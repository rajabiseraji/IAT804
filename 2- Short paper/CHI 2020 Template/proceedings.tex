\documentclass{sigchi}

% Use this section to set the ACM copyright statement (e.g. for
% preprints).  Consult the conference website for the camera-ready
% copyright statement.

% Copyright
%\CopyrightYear{2020}
%\setcopyright{acmcopyright}
%\setcopyright{acmlicensed}
%\setcopyright{rightsretained}
%\setcopyright{usgov}
%\setcopyright{usgovmixed}
%\setcopyright{cagov}
%\setcopyright{cagovmixed}
% DOI
%\doi{https://doi.org/10.1145/3313831.XXXXXXX}
% ISBN
%\isbn{978-1-4503-6708-0/20/04}
%Conference
%\conferenceinfo{CHI'20,}{April  25--30, 2020, Honolulu, HI, USA}
%Price
%\acmPrice{\$15.00}

% Use this command to override the default ACM copyright statement
% (e.g. for preprints).  Consult the conference website for the
% camera-ready copyright statement.

%% HOW TO OVERRIDE THE DEFAULT COPYRIGHT STRIP --
%% Please note you need to make sure the copy for your specific
%% license is used here!
% \toappear{
 %Permission to make digital or hard copies of all or part of this work
 %for personal or classroom use is granted without fee provided that
 %copies are not made or distributed for profit or commercial advantage
 %and that copies bear this notice and the full citation on the first
 %page. Copyrights for components of this work owned by others than ACM
 %must be honored. Abstracting with credit is permitted. To copy
 %otherwise, or republish, to post on servers or to redistribute to
 %lists, requires prior specific permission and/or a fee. Request
 %permissions from \href{mailto:Permissions@acm.org}{Permissions@acm.org}. \\
 %\emph{CHI '16},  May 07--12, 2016, San Jose, CA, USA \\
 %ACM xxx-x-xxxx-xxxx-x/xx/xx\ldots \$15.00 \\
 %DOI: \url{http://dx.doi.org/xx.xxxx/xxxxxxx.xxxxxxx}
% }

% Arabic page numbers for submission.  Remove this line to eliminate
% page numbers for the camera ready copy
% \pagenumbering{arabic}

% Load basic packages
\usepackage{balance}       % to better equalize the last page
\usepackage{graphics}      % for EPS, load graphicx instead 
\usepackage[T1]{fontenc}   % for umlauts and other diaeresis
\usepackage{txfonts}
\usepackage{mathptmx}
\usepackage[pdflang={en-US},pdftex]{hyperref}
\usepackage{color}
\usepackage{booktabs}
\usepackage{textcomp}

% Some optional stuff you might like/need.
\usepackage{microtype}        % Improved Tracking and Kerning
% \usepackage[all]{hypcap}    % Fixes bug in hyperref caption linking
\usepackage{ccicons}          % Cite your images correctly!
% \usepackage[utf8]{inputenc} % for a UTF8 editor only

% If you want to use todo notes, marginpars etc. during creation of
% your draft document, you have to enable the "chi_draft" option for
% the document class. To do this, change the very first line to:
% "\documentclass[chi_draft]{sigchi}". You can then place todo notes
% by using the "\todo{...}"  command. Make sure to disable the draft
% option again before submitting your final document.
\usepackage{todonotes}

% Paper metadata (use plain text, for PDF inclusion and later
% re-using, if desired).  Use \emtpyauthor when submitting for review
% so you remain anonymous.
\def\plaintitle{IAT804 Assignment 2: Short Paper Assessment}
\def\plainauthor{Mohammad Rajabi Seraji}
\def\emptyauthor{}
\def\plainkeywords{Research Methodology; Research Designs; Assessment; HCI; Intuitive Interaction; Natural User Interfaces}
\def\plaingeneralterms{Assessment}

% llt: Define a global style for URLs, rather that the default one
\makeatletter
\def\url@leostyle{%
  \@ifundefined{selectfont}{
    \def\UrlFont{\sf}
  }{
    \def\UrlFont{\small\bf\ttfamily}
  }}
\makeatother
\urlstyle{leo}

% To make various LaTeX processors do the right thing with page size.
\def\pprw{8.5in}
\def\pprh{11in}
\special{papersize=\pprw,\pprh}
\setlength{\paperwidth}{\pprw}
\setlength{\paperheight}{\pprh}
\setlength{\pdfpagewidth}{\pprw}
\setlength{\pdfpageheight}{\pprh}

% Make sure hyperref comes last of your loaded packages, to give it a
% fighting chance of not being over-written, since its job is to
% redefine many LaTeX commands.
\definecolor{linkColor}{RGB}{6,125,233}
\hypersetup{%
  pdftitle={\plaintitle},
% Use \plainauthor for final version.
%  pdfauthor={\plainauthor},
  pdfauthor={\emptyauthor},
  pdfkeywords={\plainkeywords},
  pdfdisplaydoctitle=true, % For Accessibility
  bookmarksnumbered,
  pdfstartview={FitH},
  colorlinks,
  citecolor=black,
  filecolor=black,
  linkcolor=black,
  urlcolor=linkColor,
  breaklinks=true,
  hypertexnames=false
}

% create a shortcut to typeset table headings
% \newcommand\tabhead[1]{\small\textbf{#1}}

% End of preamble. Here it comes the document.
\begin{document}

\title{\plaintitle}

\numberofauthors{3}
\author{%
  \alignauthor{Mohammad Rajabi Seraji\\
    \affaddr{Vancouver, Canada}\\
    \email{mrajabis@sfu.ca}}\\
}

\maketitle

\begin{abstract}
  The goal of this document is to  identify and analyze the methodological approaches, research designs, data collection and analysis methods of \cite{10.1093/iwc/iwv003} in less than 1600 words. It serves as a report for IAT804 course of SIAT SFU. \emph{\textcolor{red}{TODO: summarize all in here}}
\end{abstract}


% ACM Classfication

\begin{CCSXML}
<ccs2012>
<concept>
<concept_id>10003120.10003121</concept_id>
<concept_desc>Human-centered computing~Human computer interaction (HCI)</concept_desc>
<concept_significance>500</concept_significance>
</concept>
<concept>
<concept_id>10003120.10003121.10003125.10011752</concept_id>
<concept_desc>Human-centered computing~Haptic devices</concept_desc>
<concept_significance>300</concept_significance>
</concept>
<concept>
<concept_id>10003120.10003121.10003122.10003334</concept_id>
<concept_desc>Human-centered computing~User studies</concept_desc>
<concept_significance>100</concept_significance>
</concept>
</ccs2012>
\end{CCSXML}

\ccsdesc[500]{Human-centered computing~Human computer interaction (HCI)}
\ccsdesc[300]{Human-centered computing~Haptic devices}
\ccsdesc[100]{Human-centered computing~User studies}

% Author Keywords
\keywords{\plainkeywords}

\section{World-view}
The abstract of the document literally commences with a statement about addressing the issues and challenges of natural user interface designers\cite{10.1093/iwc/iwv003}. The positioning of these statements alongside the essence of conclusion section shows us that the authors are following a "\textbf{Pragmatic World-view}". 

\section{Research Design and Methodology}
It is explicitly stated in the abstract that \cite{10.1093/iwc/iwv003} uses an "\textbf{Exploratory mixed method}" research design. Reviewing the text also supports this claim about; they conduct a series quantitative experiments followed by different open-ended reviews and questionnaires. Their method of data collection is also a mixture of qualitative and quantitative methods. They are also both interested in answering the qualitative question of how to improve designers experience in natural user interfaces, as well as exploring the quantitative statistical comparisons of usability, intuitive interactions and awareness. Therefore we can safely conclude that this study is using a "\textbf{Mixed method design}" with an "\textbf{Exploratory comparative methodology}." It also uses a "\textbf{Between-subjects experiment design}" to conduct the experiments in the quantitative phase. 

\section{Data Collection Methods}
This study collects quantitative data in a series of between-subject \textbf{experiments} (task completions) and collects task-completion times from them. It also uses \textbf{questionnaires} between some of these tasks to obtain Likert values for responses. Basically it uses different scales and established measurement systems to quantify the outcomes of each experiment. 

There is also qualitative data collection in form of \textbf{interviews} with open-ended questions, \textbf{observational notes} and \textbf{video recordings} from the experiment sessions.

\section{Use of Theory}
This is one of the most interesting aspects of this study. The study begins by putting forward three hypotheses at the end of introduction section. These hypotheses are theories based on previous works and elicit the sense of a "\textbf{deductive use of theory/ going in}" in a qualitative manner. They follow it up by conducting experiments to test their hypotheses and report the statistical results in both discussion and conclusion sections. All of these follow a regular pattern of deductive reasoning for using a theory to test it. But, the results of the qualitative parts (open-ended interviews and questions) introduce new, unexpected patterns to the researchers; ones that they use later in the conclusion section to put out theories about user satisfaction, intuitive interaction and usability\cite{10.1093/iwc/iwv003}. And, in that sense the study introduces new knowledge in form of guidelines and theories about these variables therefore can \textbf{also} be counted as an "\textbf{inductive / going out with creation of a theory}". 

\section{Validity}
There are some internal threats to the validity of this study. It is mentioned that the majority of the participants (seventy-two percent) were chosen from the university students \cite{10.1093/iwc/iwv003}. It is also mentioned that most of the participants were daily users of computers and smart-phones. All of these factors introduce an internal validity threat in form of \textbf{selection of participants}. This threat has not been properly addressed in the paper. Unfortunately this distribution can also result in an \textbf{external interaction of selection and treatment threat} which may prevent the study from being generalized to a wider population.

Another internal threat could be \textbf{Testing} threat in which the familiarization of the participants with the instrument may affect the outcomes\cite{creswell2018research}. Fortunately by proper timing of the tasks and interviews and also sequentializing the nature of the experiments, researchers have addressed this issue in an ample manner.

\section{Ethical Concerns}
Since some of the participants were from university students, there might be an issue of power relationship of undue effect. For instance the graduate students could have been friends of the research staff or students studying under the staff's tutorship. All of these could have introduced a certain bias into the study. Although, since the CHI requires (or recommends) ethical reviews from REBs before accepting a submission, these concerns might have been properly addressed in the ethics proposal.

\section{Reasons behind choosing approach}
This is discussed in each section regarding the world-view, methodology and method. But, here is a summary. 
\subsection{World-view}
it is actively trying to address an issue in NUI designers' community therefore a pragmatic philosophical view of the world suits it the best. 
\subsection{Design and Methodology}
The authors are going to provide guidelines for a community and achieving this goal needs statistical analysis of different measures for scientific comparison, but it also needs open-ended and qualitative questions that ask about experience of people in working with set system. So, the researchers are accepting the shortcomings and benefits of each design hence adopting a mixed method design with an exploratory, comparative methodology. The use of this methodology is justified by the outcome of the study; since they want answers to a qualitative question but they need to analyze statistical data and devise measures, they are going with this particular methodology. 

Usage of experimental between-subject methodology for quantitative parts lets the researchers isolate the effect of the independent variables upon a dependent variables. It also permits the experimenter to build up models
of interactions among variables to better understand the differential influence of a variable across a range of others\cite{Gergle2014}.
\subsection{Methods}
Using experiment makes sense because they seek to know the difference between certain variables to discover a better solution. The choice of a between-subject experiment is to avoid learning and carry-over effects between subjects\cite{Gergle2014} Usage of questionnaires and open-ended reviews are also certain characteristics of the qualitative parts of their study in which they need to know about the user's feeling and satisfaction from their performance, something that can be described easier in terms of sentences, rather than quantified terms.

\section{The Relation of Knowledge Claim and Approach}
The knowledge claims of this study are comprised of 1) confirmation/denial of past theories (ones that formed the hypotheses) and 2) Qualitative results and guidelines for enhancing the NUI designers and users. Judging on the nature of the approach (mixed method) these claims are all expected and justified. The quantitative side of the study has provided enough evidence to confirm or refute the hypotheses. The combination of these approaches has also created a better insight into users' satisfaction and intuitive interaction with NUIs, which is indeed the expected result of a mixed method study.

\section{Other Approaches}
The main goal of this research is to find out how to overcome the difficulties in the design of natural user interfaces which on the first glance might look like a problem that can be resolved by careful quantitative experiments, fine instrumentation and a large-enough participant pool. Yet, when we look at the design section we find out that the most prominent results are the ones coming from the qualitative parts of this research such as interviews and open-ended questions. So the usage of a mixed-method approach may seem justified. But, we cannot overlook the fact that if the measurement instrument (Springboard) had worked in a better way or if the number of participants was high enough, this study could be done as a pure quantitative between-subject experiment with different results or better justifications for lack of difference in data. 

The main reason for opting for a quantitative approach is that based on the works in literature and this paper, all the needed variables could be perfectly quantified, measured and analyzed in an empirical manner which must have adhered to more strict scientific standards and therefore could have produced better results. 

% BALANCE COLUMNS
\balance{}

% REFERENCES FORMAT
% References must be the same font size as other body text.
\bibliographystyle{SIGCHI-Reference-Format}
\bibliography{sample}

\end{document}

%%% Local Variables:
%%% mode: latex
%%% TeX-master: t
%%% End:
